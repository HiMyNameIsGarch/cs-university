\documentclass{article}
\usepackage{graphicx} % Required for inserting images
\usepackage{amsmath}
\usepackage{colortbl} % For cell coloring
\usepackage{amssymb} % For additional symbols

\title{Homework 2, Section B}
\author{Gabriel Hanu}
\date{31 March 2025 \\
Numerical Analysis}


\begin{document}

\maketitle

\section*{Problem 1}
Construct the polynomial that interpolates the data below and approximate f(2):
\begin{center}
    \begin{tabular}{c| c c c}
        x & -2 & 1 & 3 \\
        \hline
        f(x) & 1 & 0 & 2 \\
    \end{tabular}
\end{center}

For this problem, we will use the Lagrange polynomial is the classical form:
\[
    (L_mf)(x) = \sum_{i=0}^{m} l_i(x)f(x)
\]
where the Lagrange basis polynomials \(l_i(x)\) are defined as:
\[
    l_i(x) = \prod_{j=0, j\neq i}^{m} \frac{x-x_j}{x_i-x_j}
\]

Before computing all the Lagrange basis polynomials, we can analyze and see
that \(l_1(x)\) is mutiplied by 0, so we can simplify the expression and
compute only \(l_0(x)\) and \(l_2(x)\).

\[
    l_0(x) = \frac{(x-1)(x-3)}{(-2-1)(-2-3)} = \frac{(x-1)(x-3)}{15}
\]
\[
    l_2(x) = \frac{(x+2)(x-1)}{(3+2)(3-1)} = \frac{(x+2)(x-1)}{10}
\]

Now we can compute the Lagrange polynomial:
\[
    (L_2f)(x) = l_0(x)f(-2) + l_1(x)f(1) + l_2(x)f(3)
\]
\[
    (L_2f)(x) = \frac{(x-1)(x-3)}{15} \cdot 1 + 0 \cdot 0 + \frac{(x+2)(x-1)}{10} \cdot 2
\]

\[
    (L_2f)(x) = \frac{(x-1)(x-3)}{15} + \frac{(x+2)(x-1)}{5}
\]
\[
    (L_2f)(x) = \frac{(x-1)(x-3) + 3(x+2)(x-1)}{15}
\]
\[
    (L_2f)(x) = \frac{4x^2-x-3}{15}
\]

Now we can evaluate the polynomial at x=2:
\[
    (L_2f)(2) = \frac{2^2 - 2 - 3}{15}
\]
So the final result is:
\[
    (L_2f)(2) = \frac{11}{15}
\]

\section*{Problem 2}
Using Newton’s form, obtain the polynomial that interpolates the following data:
\begin{center}
    \begin{tabular}{c| c c c c c c}
        x & -2 & -1 & 0 & 1 & 2 & 3 \\
        \hline
        f(x) & 31 & 5 & 1 & 1 & 11 & 61 \\
    \end{tabular}
\end{center}

In order to solve this problem, we will use the Newton's form of the interpolation polynomial:
\[
    (N_mf)(x) = f(x_0) + \sum_{i=1}^{m} (x-x_0)...(x-x_{i-1}) \cdot (D^if)(x_0)
\]
Knowing the order m = 5, we can compute the divided differences:
\begin{center}
    \begin{tabular}{c c | c c c c c}
        x & f(x) & \(D'(x)\) & \(D''(x)\) & \(D'''(x)\) & \(D^4(x)\) & \(D^5(x)\) \\
        \hline
        -2 & 31 & -26 & 11 & -3 & 1 & 0 \\
        -1 & 5 & -4 & 2 & 1 & 1 & - \\
        0 & 1 & 0  & 5 & 5 & - & - \\
        1 & 1 & 10 & 20 & - & - & - \\
        2 & 11 & 50 & - & - & - & - \\
        3 & 61 & - & - & - & - & - \\
    \end{tabular}
\end{center}

We can now compute the Newton's form of the polynomial:
\begin{align*}
(N_5f)(x) = 31 & + (x+2)D'(x) \\
         & + (x+2)(x+1)D''(x) \\
         & + (x+2)(x+1)xD'''(x) \\
         & + (x+2)(x+1)x(x-1)D^4(x) \\
         & + (x+2)(x+1)x(x-1)(x-2)D^5(x) \\
\end{align*}
but in this case \(D^4(x)\) = 1 and \(D^5(x)\) = 0, so we can simplify the expression:

\begin{align*}
(N_5f)(x) = 31 & + (x+2)(-26) \\
         & + (x+2)(x+1)(11) \\
         & + (x+2)(x+1)x(-3) \\
         & + (x+2)(x+1)x(x-1) \\
\end{align*}
Finally, by doing the computations we get:

\[
    (N_4f)(x) = x^4-x^3+x^2-x+1
\]


\section*{Problem 3}
Find the maximum approximation error of \(cos(\frac{\pi}{4})\) using the Lagrange polynomial corresponding
to \(f(x)=cos(\pi x)\) and the nodes \(x_0=0,\ x_1=\frac{1}{3},\ x_2=\frac{1}{2} \).
\newline
\newline
Lagrange polynomial in the classical form is defined by:
\[
    f(x) = L_mf(x) + R_mf(x)
\]
with \(R_mf(x) = \frac{u(x)}{(m+1)!}f^{m+1}(\zeta)\), where
\[
    u(x) = \prod_{j=0}^{m} (x-x_j)
\]
and
\[
    \zeta \in [\alpha,\beta] \text{ where } \alpha = \min\{x_0,...,x_m\} \text{ and } \beta = \max\{x_0,...,x_m\}
\]
We begin by computing \(u(x)\):
\[
    u(x) = \big(x-0\big)\bigg(x-\frac{1}{3}\bigg)\bigg(x-\frac{1}{2}\bigg)
\]
Now we need to compute the third derivative of \(f(x)\):
\[
    f(x) = cos(\pi x); \
    f'(x) = -\pi sin(\pi x); \
\]
\[
    f''(x) = -\pi^2 cos(\pi x); \
    f'''(x) = \pi^3 sin(\pi x); \
\]

For the interval of \(\zeta\) we can see that \(\alpha\) = 0 and \(\beta\) = \(\frac{1}{2}\), so it is clearly that:
\[
    \zeta \in \bigg[0,\frac{1}{2}\bigg]
\]
Putting it all together we get:

\[
    R_2f(x) = \frac{x(x - \frac{1}{3})(x-\frac{1}{2})}{3!}
    \cdot \pi^3 sin(\pi \zeta)\text{, with } \zeta \in \bigg[0,\frac{1}{2}\bigg]
\]

On the interval \(\zeta \in [0,\frac{1}{2}]\), the maximum value of the sine function is 1. Using x = \(\frac{\pi}{4}\) we get:

\[
    R_2f(x) = \frac{1}{192} \cdot \frac{\pi^3}{6}
\]

Finally, the maximum approximation error is:
\[
    R_2f(x) =~ 0.0269
\]


\section*{Problem 4}
Approximate ln(43) if ln(2) = 0.6931, ln(3)=1.0986, ln(5) = 1.6094
\newline

Firstly we can try to approximate ln(43) using the Lagrange polynomial, so we can write the table:

\begin{center}
    \begin{tabular}{c| c c c}
        x & 2 & 3 & 5 \\
        \hline
        ln(x) & 0.6931 & 1.0986 & 1.6094 \\
    \end{tabular}
\end{center}

We already know the formulas so we will do just the computations:
\[
    l_0(x) = \frac{(x-3)(x-5)}{3} \, \
    l_1(x) = \frac{(x-2)(x-5)}{-2} \ \
    l_2(x) = \frac{(x-2)(x-3)}{6}
\]

So we can write the Lagrange polynomial:
\[
    (L_2f)(x) = \frac{(x-3)(x-5)}{3} \cdot 0.6931 + \frac{(x-2)(x-5)}{-2} \cdot 1.0986 + \frac{(x-2)(x-3)}{6} \cdot 1.6094
\]
Plug in the value we try to approximate: \(x=43\)
\[
    (L_2f)(43) = \frac{40 \cdot 38}{3} \cdot 0.6931 + \frac{41 \cdot 38}{-2} \cdot 1.0986 + \frac{41 \cdot 40}{6} \cdot 1.6094
\]
And we get:
\[
    (L_2f)(43) = -64.67616652
\]
I didn't expect this result, but my guess is that the Lagrange polynomial
works best when the nodes are close to the point we try to approximate. The
point 43 is too far from the data range and the polynomials amplifies the
error.
\newline
Ok, I think we can use the rules of logarithms to solve this problem. If we
analize the number 43 we can see that is a prime therefore we can't factorize
it.
\newline
But, we already know the values of log in 2, 3 and 5, we can deduce that ln(43) lies somewhere between ln(45) and ln(40).
Here we can express 40 as \(2^3 \cdot 5\) and 45 as \(3^2 \cdot 5\), so we can write:
\[
    ln(40) = ln(2^3 \cdot 5) = 3 \cdot ln(2) + ln(5) = 3 \cdot 0.6931 + 1.6094
\]
\[
    ln(40) =~ 3.6887
\]
and
\[
    ln(45) = ln(3^2 \cdot 5) = 2 \cdot ln(3) + ln(5) = 2 \cdot 1.0986 + 1.6094
\]
\[
    ln(45) = 3.8066
\]
So ln(43) lies between these 2 numbers. But to give a concise answer,
we can say that \(ln(43) = ln(40) + \frac{3}{5} \cdot (ln(45) - ln(40))\) = 3.7594
\newline
\newline
I hope that the second method is correct and it really solves the problem proposed.

\end{document}
