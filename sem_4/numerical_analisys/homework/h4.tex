\documentclass{article}
\usepackage{amsmath}
\usepackage{amsfonts}
\usepackage{amssymb}

\title{Homework 4, Section B}
\author{Hanu Gabriel}
\date{5 May 2025 \\
Numerical Analysis}

\begin{document}

\maketitle

\section*{Problem 1}
Check if the following function is a natural cubic spline on the interval [1, 2]

\[
S(x) =
\begin{cases}
x e^{t-1}, & x \in [-1, 0] \\
1 + 2(x+1) + (x+1)^3 + 5x + 3x, & x \in [0,1] \\
1 + (x-1) + 3(x-2)^2 + (x-1)^3, & x \in [1,2]
\end{cases}
\]

A function \( S(x) \) is a natural cubic spline on \([-1,2]\) if \( S, S', S'' \) are continuous and:
\[
S''(-1) = 0, \quad S''(2) = 0
\]

Compute derivatives:
\[
S'(x) =
\begin{cases}
2 + 3(x+1)^2, & x \in [-1,0] \\
5 + 6x, & x \in [0,1] \\
1 + 6(x-2) + 3(x-1)^2, & x \in [1,2]
\end{cases}
\]
\[
S''(x) =
\begin{cases}
6(x+1), & x \in [-1,0] \\
6, & x \in [0,1] \\
6 + 6(x-1), & x \in [1,2]
\end{cases}
\]

Check continuity at \( x = 0 \) and \( x = 1 \):

At \( x = 0 \):
\[
S(0^-) = 1 + 2 + 1 = 4, \quad S(0^+) = 3 \Rightarrow S(x) \text{ is not continuous at } x = 0
\]

Therefore, \( S(x) \) is \textbf{not} a natural cubic spline on \([-1,2]\).

\section*{Problem 2}
Determine coefficients \( b, c, d \) for a natural cubic spline

\[
S(x) =
\begin{cases}
1 + 2x - x^3, & x \in [0,1] \\
2 + b(x-1) + c(x-1)^2 + d(x-1)^3, & x \in [1,2]
\end{cases}
\]

The spline must satisfy continuity of \( S, S', S'' \), and:
\[
S''(0) = 0, \quad S''(2) = 0
\]

Compute derivatives:

\[
S'(x) =
\begin{cases}
2 - 3x^2, & x \in [0,1] \\
b + 2c(x-1) + 3d(x-1)^2, & x \in [1,2]
\end{cases}
\]

\[
S''(x) =
\begin{cases}
-6x, & x \in [0,1] \\
2c + 6d(x-1), & x \in [1,2]
\end{cases}
\]

Continuity conditions:
\begin{align*}
S_0(1) &= S_1(1) &&\Rightarrow 2 = 2 \quad \text{ok} \\
S_0'(1) &= S_1'(1) &&\Rightarrow -1 = b \Rightarrow b = -1 \\
S_0''(1) &= S_1''(1) &&\Rightarrow -6 = 2c \Rightarrow c = -3 \\
S''(2) &= 0 &&\Rightarrow 2c + 6d = 0 \Rightarrow -6 + 6d = 0 \Rightarrow d = 1
\end{align*}

\textbf{Final coefficients:} \( b = -1, c = -3, d = 1 \)

\section*{Problem 3}
Fit a constant, linear, and quadratic polynomial to data

Given data points:

\[
\begin{array}{c|cccc}
x & 0 & 1 & 2 & 3 \\
\hline
y & 5 & -6 & 7 & 4 \\
\end{array}
\]


\subsection*{Constant Fit}

We compute the average of the \( y \)-values:
\[
a_0 = \frac{1}{4}(5 + (-6) + 7 + 4) = \frac{10}{4} = 2.5
\]
So, the best constant fit is:
\[
p(x) = 2.5
\]
%
\subsection*{Linear Fit}
%
We want to find \( p(x) = a_0 + a_1 x \) using the normal equations.

\begin{align*}
\sum x_i &= 0 + 1 + 2 + 3 = 6 \\
\sum y_i &= 5 - 6 + 7 + 4 = 10 \\
\sum x_i^2 &= 0^2 + 1^2 + 2^2 + 3^2 = 14 \\
\sum x_i y_i &= (0)(5) + (1)(-6) + (2)(7) + (3)(4) = -6 + 14 + 12 = 20
\end{align*}
%
The normal equations are:

\[
\begin{cases}
4a_0 + 6a_1 = 10 \\
6a_0 + 14a_1 = 20
\end{cases}
\]
%
Solving:
\[
a_1 = 1, \quad a_0 = 1
\]

So the linear fit is:
\[
p(x) = 1 + x
\]

\subsection*{Quadratic Fit}

We want to find \( p(x) = a_0 + a_1 x + a_2 x^2 \) using the normal equations.

\begin{align*}
\sum x_i^3 &= 0^3 + 1^3 + 2^3 + 3^3 = 36 \\
\sum x_i^4 &= 0^4 + 1^4 + 2^4 + 3^4 = 98 \\
\sum x_i^2 y_i &= 0^2(5) + 1^2(-6) + 2^2(7) + 3^2(4) = -6 + 28 + 36 = 58
\end{align*}

The system becomes:

\[
\begin{cases}
10 = 4a_0 + 6a_1 + 14a_2 \\
20 = 6a_0 + 14a_1 + 36a_2 \\
58 = 14a_0 + 36a_1 + 98a_2
\end{cases}
\]
%
Solving gives:
\[
a_0 = 3, \quad a_1 = -5, \quad a_2 = 2
\]

So the quadratic fit is:
\[
p(x) = 3 - 5x + 2x^2
\]

\end{document}
