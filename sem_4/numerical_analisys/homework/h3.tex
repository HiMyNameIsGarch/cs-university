\documentclass{article}
\usepackage{graphicx} % Required for inserting images
\usepackage{amsmath}
\usepackage{colortbl} % For cell coloring
\usepackage{amssymb} % For additional symbols

\title{Homework 3, Section B}
\author{Gabriel Hanu}
\date{14 April 2025 \\
Numerical Analysis}


\begin{document}

\maketitle

\section*{Problem 1}
Construct the interpolation polynomial P corresponding to the information below. Specify the type of interpolation.

\section*{Section a)}
\begin{center}
    \begin{tabular}{c| c c c}
        \(x_k\) & 0 & 1 & 2 \\
        \hline
        \(y_k\) & 1 & 3 & 21 \\
        \hline
        \(y'_k\) & 0 & 3 & 36 \\
    \end{tabular}
\end{center}
We choose to use the Hermite interpolation method with double nodes
The formula is:
\[
    H_mf(x) = f(z_0) + \sum_{i=1}^{m} (x - z_0) ... (x-z_{i-i}) D^if(z_0)
\]
Also we need to compute the divided differences table:
\begin{center}
    \begin{tabular}{c | c c c c c c}
        \(x\) & \(y\) & \(D^1y\) & \(D^2y\) & \(D^3y\)& \(D^4y\)& \(D^5y\)\\
        \hline
        0 & 1 & 0 & 2 & -1 & 4 & -3 \\
        0 & 1 & 2 & 1 & 7 & -2 & - \\
        1 & 3 & 3 & 15 & 3 & - & - \\
        1 & 3 & 18 & 18 & - & - & - \\
        2 & 21 & 36 & - & - & - & - \\
        2 & 21 & - & - & - & - & - \\
    \end{tabular}
\end{center}
Finally the Hermite interpolation polynomial is:
\[
    H_5f(x) = 1 + 0(x-0) + 2x^2 - x^2(x-1) + 4x^2(x-1)^2 -3x^2(x-1)^2(x-2)
\]
Finally we get our polynomial:
\[
    P(x) = -3x^5 + 16x^4 - 24x^3 + 13x^2 + 1
\]


\section*{Section b)}
\[
    P''(-2) = f''(-2), \
    P'(-1) = f'(-1), \
    P(0) = f(0)
\]

We will use Birkhoff interpolation to construct the minimal-degree polynomial. Since we have three conditions, a quadratic polynomial is written as:

\[ P(x) = ax^2 + bx + c \]

Compute its derivatives:
\begin{align*}
    P'(x) &= 2ax + b \\
    P''(x) &= 2a
\end{align*}

Apply the conditions:
\begin{enumerate}
    \item Second derivative at \( x = -2 \):
    \[ P''(-2) = 2a = f''(-2) \implies a = \frac{f''(-2)}{2} \]

    \item First derivative at \( x = -1 \):
    \[ P'(-1) = -2a + b = f'(-1) \implies b = f'(-1) + 2a \]
    Substituting \( a \):
    \[ b = f'(-1) + f''(-2) \]

    \item Function value at \( x = 0 \):
    \[ P(0) = c = f(0) \]
\end{enumerate}

The final polynomial is:
\[ P(x) = \frac{f''(-2)}{2}x^2 + \left(f'(-1) + f''(-2)\right)x + f(0) \]

\section*{Section c)}
\begin{center}
    \begin{tabular}{c| c c c c}
        \(x_k\) & 1 & 2 & 4 & 6 \\
        \hline
        \(y_k\) & 14 & 15 & 5 & 9 \\
    \end{tabular}
\end{center}
We choose to solve the problem using Newton interpolation with divided differences table:

The formula is:
\[
    N_n(x) = f(x_0) + \sum_{i=1}^{n} (x - x_0) ... (x-x_{i-1}) D^if(x_0)
\]
Also we need to compute the divided differences table:
\begin{center}
    \begin{tabular}{c| c c c c}
        x & y & \(D^1y\) & \(D^2y\) & \(D^3y\) \\
        \hline
        1 & 14 & 1 & -2 & 3/4 \\
        2 & 15 & -5 & 7/4 &  -\\
        4 & 5 & 2 & - & - \\
        6 & 9 & - & - & - \\
    \end{tabular}
\end{center}

So the Newton interpolation polynomial is:
\[
    N_3f(x) = 14 + 1(x-1) + (-2)(x-1)(x-2) + \frac{3}{4}(x-1)(x-2)(x-4)
\]
Reducing the polynomial we get:
\[
    P(x) = \frac{3}{4}x^3 - \frac{21}{4}x^2 - 2x^2 + 7x + \frac{21}{2}x - 15
\]


\section*{Section d)}
\begin{center}
    \(x_0 = 0\) simple node, \(x_1 = 1\) triple node
\end{center}
We may proceed to compute the interpolation polynomial using the Hermite interpolation method
Using the formula:
\[
    H_mf(x) = \sum_{k=0}^{m} \sum_{j=0}^{r_k} h_{kj}(x)f^{(j)}(x_k)
\]

We will need only these polynomials:
\[
    h_{00}(x), h_{10}(x), h_{11}(x), h_{12}(x)
\]
with the general formula for \(h_{kj}(x)\) being:
\[
    h_{kj}(x) = ax^3 + bx^2 + cx + d
\]
\[
    h'_{kj}(x) = 3ax^2 + 2bx + c
\]
So P(x) will be of the form:
\[
    P(x) = h_{00}(x)f(x_0) + h_{10}(x)f'(x_1) + h_{11}(x)f'(x_1) + h_{12}(x)f''(x_1)
\]
Now we compute the polynomials:

\[
\begin{array}{r@{}l@{\quad}r@{}l@{\quad}r@{}l}
    h_{00}(x_0) &{}= 1 & a + b + c + d &{}= 0 & a &{}= -1\\
    h_{00}(x_1) &{}= 0 & d &{}= 1 & b &{}= 3,\  \\
    h'_{00}(x_1) &{}= 0 & c &{}= 1 & c &{}= -3,\ \\
    h_{00}(x_2) &{}= 0 & a + b + c + d &{}= 0 & d &{}= 1 \\
\end{array}
\]

Finally it is:
\[
    h_{00}(x) = -x^3 + 3x^2 - 3x + 1
\]

\[
\begin{array}{r@{}l@{\quad}r@{}l@{\quad}r@{}l}
    h_{10}(x_0) &{}= 0 & a + b + c + d &{}= 1 & a &{}= 1\\
    h_{10}(x_1) &{}= 1 & d &{}= 1 & b &{}= -3,\  \\
    h'_{10}(x_1) &{}= 0 & c &{}= 1 & c &{}= 3,\ \\
    h_{10}(x_2) &{}= 0 & a + b + c + d &{}= 1 & d &{}= 0 \\
\end{array}
\]

Finally it is:
\[
    h_{10}(x) = x^3 + -3x^2 + 3x
\]

\[
\begin{array}{r@{}l@{\quad}r@{}l@{\quad}r@{}l}
    h_{11}(x_0) &{}= 0 & a + b + c + d &{}= 1 & a &{}= -1\\
    h_{11}(x_1) &{}= 0 & d &{}= 1 & b &{}= 3,\  \\
    h'_{11}(x_1) &{}= 1 & c &{}= 1 & c &{}= -2,\ \\
    h_{11}(x_2) &{}= 0 & a + b + c + d &{}= 1 & d &{}= 0 \\
\end{array}
\]

Finally it is:
\[
    h_{11}(x) = -x^3 + 3x^2 - 2x
\]

\[
\begin{array}{r@{}l@{\quad}r@{}l@{\quad}r@{}l}
    h_{12}(x_0) &{}= 0 & a + b + c + d &{}= 1 & a &{}= 1/2\\
    h_{12}(x_1) &{}= 0 & d &{}= 1 & b &{}= -1,\  \\
    h'_{12}(x_1) &{}= 0 & c &{}= 1 & c &{}= 1/2,\ \\
    h_{12}(x_2) &{}= 1 & a + b + c + d &{}= 1 & d &{}= 0 \\
\end{array}
\]

Finally it is:
\[
    h_{12}(x) = \frac{1}{2}x^3 - x^2 + \frac{1}{2}x
\]
Putting it all together we get the following:
\[
    P(x) = \big( -x^3 + 3x^2 - 3x + 1 \big)f(x_0) + \big(x^3 + -3x^2 + 3x\big)f'(x_1)
\]
\[
    + \big(-x^3 + 3x^2 - 2x\big)f'(x_1) + \big(\frac{1}{2}x^3 - x^2 + \frac{1}{2}x\big)f''(x_1)
\]



\end{document}
